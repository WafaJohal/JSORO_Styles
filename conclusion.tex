\section{Conclusion}
\label{sec:conclusion}
%original hypothesis:
The need for social customization of companion's robot behaviour led us to propose behavioural styles and to assume that it was possible for users to distinguish between styles and, and that they were impacting the robot-user interaction.

%Findings
The \dots model provided realistic simulations of \dots for these two \dots and may be applicable to other types of \dots.

%Explanation for findings


%Limitations
However, the model contains 
object manipulation have not been take into account. 


%Need for further research


%futur for styles and personalisation
In the future, even the appearance should be customizable by the user since some studies have shown that there exist systematic individual differences in terms of preference of the companion's appearance \cite{Walters2008}. 
They have highlighted individual differences in the preference of the robot's dynamic which could suggest and emphasise the need for the social adaptation of robots. 
From these researches the Uncanny Valley\footnote{Uncanny Valley:} is not at the same location for each individual and the creepiness threshold might also be varying. 
