\section{Results}
\label{sec:results}
In order to analyse these various measures we decided to group them according to the style factors as defined by Gallaher \cite{Gallaher1992}. 
Gallaher propose four style factors: 
\begin{itemize}
\item Expressiveness: that describe how \dots
\item Animation :
\item Expansiveness : 
\item Coordination : 
\end{itemize}
Eventhough Gallaher presents these dimension has highly correlated to the personality, we decide to compare these dimension for each individual and between our two conditions. 
Thereby, we would be able to see the comparative effect on these dimensions on the user of our conditions. 

\begin{table}[h]
	\small
	\centering
	\begin{tabular}{|l |c| l }\hline
		\textbf{Behavioural Items}  &	\textbf{Style factor} & Measure \\ \hline
		Uses very little-most of body when gesturing & Expressiveness  & \\
		Stow-fast gestures  & Expressiveness  & speed of motion\\
		Gestures: infrequently-frequently & Expressiveness & QoM\\
		Shakes head: frequently-rarely & Expressiveness & \\
		Narrow-broad gestures & Expressiveness & body volume\\
		Nods head: frequently-rarely & Expressiveness & head nods\\ \hline
	
		Shoulders: slumped-erect when standing & Animation &\\
		Sits down and stands up: slowly-quickly & Animation &\\
		Torso: upright-leaning when standing & Animation & BLA\\
		Sits with torso: tilted-vertical & Animation & BLA\\
		Slow-fast walker & Animation &\\
		Sits with torso: erect-slumped & Animation & BLA\\ \hline
	
		Legs: together-wide apart when sitting & Expansiveness & feet relationship\\
		Soft-loud voice & Expansiveness & \\
		Elbows: close to-away from body & Expansiveness & elbow distance or elbow extensiveness\\
		Thin-full voice & Expansiveness &\\
		Light-heavy step & Expansiveness &\\
		Takes: small-large steps & Expansiveness & \\
		Legs: close together-wide apart when standing  & Expansiveness & legs extensiveness\\
		Hands: close to-away from body & Expansiveness & hand distance\\ 
		Soft-loud laughter & Expansiveness & \\ \hline
		
		Choppy-rhythmic speech & Coordination&\\
		%Jerky-fluid walk & Coordination&\\
		Rough-smooth gestures & Coordination& jerkiness\\
		Harsh-smooth voice & Coordination& jerkiness\\ 
		\hline
	
	\end{tabular}
	\label{tab:styledims_gallaher}
	\caption{Table of Behaviours for each style factor from \cite{Gallaher1992}}
	
\end{table} 


\subsection{Expressiveness Measures}
\subsubsection{Quantity of Motion}
%todo show an example of QoM throuout the whole experiment with colouring of the activity



%Statistics
%ANOVA results are cited using the F test (e.g., $F (1, 38) = 4.94$, $p = .04$). F and p are italicized. Give the exact $p$ value to two decimal places except when greater better than $p < .01$. Try to avoid more than two decimal places, as readers are not very interested in exact values; comparisons are more important. The $F$ value is followed by the degrees of freedom in the numerator and denominator. Here is how one author of a mobile phone study (Author, 2010) reported an interaction effect:
%\begin{quotation}
%The data were analyzed in a 2 (Age: young vs. old) x 2 (Device: Phone A vs. B) mixed measures analysis of variance (ANOVA). We found an interaction of Age X Phone ($F [1, 38] = 26.18$, $p < .0001$). The contrast showed older people using Phone B took considerably longer to complete the task, $F (1, 38) = 69.16$, $p = .02.$ (p. 6).
%\end{quotation}
%Correlations using product moment tests are reported using $r$ (e.g., $r (200) = .19$). The number in parenthesis following the $r$ value represents the degrees of freedom ({\it df}),  or N Ð 1. Including the degrees of freedom provides the reader with a sense of statistical significance, in that correlations are highly sensitive to the number of data points. For instance, with 200 scores on both variables, .19 will be a significant correlation whereas with only 30 people, it will not be statistically significant.
%
%T tests (e.g., $t [20] = 100.2$, $p < .01$) are reported giving the degrees of freedom in the denominator. (The numerator {\it df} is always $1$, as is true of correlations.)
%Notice how brackets are placed inside parentheses to reduce confusion.

