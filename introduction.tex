\section{Introduction}
\label{sec:introduction}
%paragraph1 : general prgagraph about context 
As technology develops, there is a tendency of the multiplication of the numerical platforms in home-environments. 
The new dynamic of the Internet-Of-Things and connected objects have accelerated this phenomenon. 
%Computer devices, tablet or cellphone users haven't yet established a social relationship with them. 
%Creating a social relationship with the numerical devices could help increase trust and quality of life of individuals.
Where the users used to see these platforms as tools, they could be able to exploit these artificial entities and create a social relationship that is \emph{trustable}, \emph{controllable} and \emph{credible}. 
To this end, this user-tool relationship could evolve into a user-companion relationship, in which the user can count on his companion to care, help and entertain him.

Human-human communication is a large detailed field of study. 
Some signals sent between human interlocutors are subconscious while still having been processed by their cognition.
Subconsciously, humans emit social signals that shows their intentions and their goals.
These abilities are parts of what is called the \emph{social intelligence}.
Several research fields, from human-computer interaction, affective computing, social signal processing, human-robot interaction, human activity recognition etc. are now working on enabling new technologies with social intelligence in order to make them more ``user-friendly'' and acceptable.
Researchers in HRI~\cite{Tapus2007,Dautenhahn2007} have also noted that robots need not only to be competent in the task but also socially intelligent.
Researchers in Human-Robot Interaction (HRI) have agreed that acceptability in home environments will come only via sociability \cite{Dautenhahn2007} of the robots.


%our vision
The underlying assumption of this work is that the social personalisation of the companion's robot behaviour improves the acceptability of the users.
The interest in the personalisation of companion robots comes from the assumption that users want a more socially intelligent robot while keeping a certain controllability of their home assistant.
It is also assumed that social robots should display comprehensive behaviours for the user - in its decision and expressions.
For this, its reasoning processes should be made explicit to the user providing a level of understanding. 
Customisation of the behaviours on the other hand can improve the user's feedback as well as the controllability of his companion robot.
Our goal is to propose a way for the user to chose the manner their companion robot would behave. 
This aims to contribute in both acceptability of the companion and to make the relationship and social bounding easier with the companion. 
In particular, we investigate how the personalisation of the social robot's behaviour in context based on \emph{style} can be designed and evaluated in a realistic context and can contribute to \emph{social adaptability} of the companion robot's behaviour.

The challenge of the social adaptability of a robot's behaviour can be approached from different perspectives. 
The design of companion robots requires an understanding and a model of the \emph{plasticity} as well as reasoning and developing communicating vector of this adaptation. 
Our work takes concepts from several disciplines, aiming to benefit from various approaches of plasticity.
As the research went along, this work used notions from socio-psychology, cognitive sciences, artificial intelligence and affective reasoning, human-machine interaction, computer sciences and human-robot interaction.

From a more technical point of view, the variations in the ways social roles that a social robot  has to play can be seen as styles applied to a predefined script.
The notion of style and social role are important in our project: the robot plays a social role with a particular style in order to fit the social expectations of the user better.
A social role is understood as a set of abilities, goals and beliefs of the robot.
The style is defined by the way of performing the social role.

%could ...

%research questions
%\paragraph{Research Questions}
%\begin{itemize}[noitemsep,nolistsep]
%	\item Why should companion robots have styles ? (Section \ref{sec:soa_styles} )
%	\item What distinguishes our approach from existing approaches for personalisation? (Section \ref{sec:soa_styles})
%	\item Given the physical constraints of the robots (e.g., velocity limits, variation in mechanical degrees of freedom), how can this personalisation be carried out?(Chapter \ref{cha:stylemode})
%	\item Can styles (parental) expressed by robots be recognized by parents ? (Chapter \ref{cha:evalonline})
%	\item What is the influence of styles on the children's behaviours? (Chapter \ref{cha:stylebotxp})
%	\item How can we measure the attitude changing in an interaction with a social robot? (Section \ref{sec:soa_nv})
%	\end{itemize}

%paragraph 2 others, .... last sentence but pb
%todo fininsh intro
%paragraph 3 : In this paper we propose ...
In this paper we propose a role specific way to customize the non-verbal behaviour of robots called \emph{behavioural style}.
We present a child-robot interaction experiment aiming to investigate the influence of behavioural styles displayed by a humanoid robot on the perception and the attitude of a child in interaction with this robot. 
We also introduce some bodily metrics computed from Kinect skeleton data aiming to evaluate the children's attitudes towards each style played by the robot.

In Section~\ref{sec:relwork}, we discuss examples that motivate our work and place it in the context of other related work.
Section~\ref{sec:styles} ~introduces the concept, implementation and model validation of \emph{behavioural styles} by evaluating expressibility by two different robotic platform.
Section~\ref{sec:research-questions-and-scenario} presents the experimental scenario to evaluate \emph{behavioural styles} in a child-robot interaction. 
We introduce new measures used to evaluate child's attitude during the interaction in section~\ref{sec:measures}.
Results of this experimentation are presented in section~\ref{sec:results}.
We also discuss the applicability of behavioural styles and proposed metrics to other HRI contexts (see section ~\ref{sec:discussion}).
%Finally, we describe an additional experiment designed to assess the benefit of ... in section 8.
Finally, we conclude with recommendations for the direction of future research in Section~\ref{sec:conclusion}.

%}We introduce a way to measure the attitude of the children users.
%we compare attitudes of user according to behavioural style
%we show that styles have an effect on the user
%we also present results from experimental analysis indicating that 
%findings from ... test indicate  statistically significant differences un 
%
%additionally
%
%finally we posited that even 
%we conducted a follow up experiment to test this hypothesis
%
%
%






